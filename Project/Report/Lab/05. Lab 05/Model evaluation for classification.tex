%! Author = anton
%! Date = 30/08/2024

To assess whether one model classifies correctly or it is better than another, it is important to introduce what
may be a unit of measurement.
In particular, the evaluation of a model is done using the \textbf{DCF (Detection Cost Function)}
or also called \textbf{empirical Bayes Risk}.
But before delving into this method of valuation, we must consider what is called the working point
of a binary classification application.
This point is characterised by a triplet of values:

\begin{equation}
    \((\pi_T, C_{fn}, C_{fp})\)
    \label{eq:priorCfnCfp}
\end{equation}

where \(\pi_T\) is the prior probability of the target class, \(C_{fn}\) is the cost of a false negative and \(C_{fp}\) is the cost of a false positive.

But from \(\pi_T\), one can have attention on a particular triplet that is defined by:

\begin{equation}
    \((\tilde{\pi}, C_{fn}, C_{fp})\) = \((\tilde{\pi}, 1, 1)\)
    \label{eq:effecrivePriorCfnCfp}
\end{equation}

where \(\tilde{\pi}\) in \autoref{eq:effecrivePriorCfnCfp} is called \textbf{effective prior} probability of the target class, defined as:

\begin{equation}
    \tilde{\pi} = \frac{\pi_T C_{fn}}{\pi_T C_{fn} + (1 - \pi_T)C_{fp}}
    \label{eq:effectivePrior}
\end{equation}

So we can define \(DCF_u\) as:

\begin{equation}
    B_{emp} = DCF_u = \sum_{c=1}^{K} \frac{\pi_c}{N_c} \sum_{i\mid c_i = c} C(a(x_i, R)\mid c)
    \label{eq:empiricalBayesRisk}
\end{equation}

From \autoref{eq:empiricalBayesRisk}, we can obtain:

\begin{equation}
    DCF_u(\pi_T, C_{fn}, C_{fp}) = \pi_{T}C_{fn}P_{fn} + (1 - \pi_{T})C_{fp}P_{fp}
    \label{eq:DCFUnormalized}
\end{equation}

where:

\begin{equation}
    P_{fn} = \frac{FN}{FN + TP}\text{ ,}\quad
    P_{fp} = \frac{FP}{FP + TN}
    \label{eq:PfnPfp}
\end{equation}

From the \autoref{eq:empiricalBayesRisk}, it is possible to find \textbf{minDCF} and \textbf{actDCF}.
When calculating minDCF, the threshold used is the one that minimize DCF, and there are various methods for finding it.
Whereas for actDCF, the threshold used is:

\begin{equation}
    t' = - \log \frac{\tilde{\pi}}{1 - \tilde{\pi}}
    \label{eq:actDCFThreshold}
\end{equation}