%! Author = antonio
%! Date = 7/2/24

To perform the classification, the dataset must first be divided into two sub-portions, the training and validation sub-portions.

\subsection{Gaussian models}
\label{subsec:gaussianModels}
Since, it is dealing with a binary classification task, it will assign a probabilistic score to each sample in terms
of the class-posterior log-ratio:
\begin{equation}
    \log r(x_t) = \log \frac{P(C=h_1\mid x_t)}{P(C=h_0\mid x_t)}
    \label{eq:llr}
\end{equation}

Analysing \autoref{eq:llr} in more detail, it becomes:
\begin{equation}
    \log r(x_t) = \log \frac{f_{X\mid C}(x_t \mid h_1)}{f_{X\mid C}(x_t \mid h_0)} + \log \frac{P(C=h_1)}{P(C=h_0)}
    \label{eq:llrExpanded}
\end{equation}

The first addend of the equation is called the \textit{llr} or \textit{log-likelihood ratio} and an optimal decision is
given by \autoref{eq:llrDecision}.

\begin{equation}
    \log r(x_t) \gtrless 0
    \label{eq:llrDecision}
\end{equation}

Considering \(P(C=h_1) = \pi \) and \(P(C=h_0) = 1 - \pi\), from \autoref{eq:llrExpanded} and \autoref{eq:llrDecision},
it is possible to write that the class assignment is based on \autoref{eq:llrExpanded} and \autoref{eq:llrDecision},
to obtain \autoref{eq:assignmentClasses}.

\begin{equation}
    llr(x_t)=\log \frac{f_{X\mid C}(x_t \mid h_1)}{f_{X\mid C}(x_t \mid h_0)} \gtrless -\log \frac{\pi}{1 - \pi}
    \label{eq:assignmentClasses}
\end{equation}

The optimal class decision is based on a comparison between the \(\llr\) and a threshold \(\th\),
if the llr is greater than th the sample is assigned to class \(h_1\), otherwise to class \(h_0\).
It is necessary to find the parameters \(\theta\), \(\mu_c\), \(\Sigma_c\); this can be done by maximising the log-likelihood.
Parameter estimation is part of the training phase and this therefore performed on the training part of the dataset,
then an estimation of the error rate can be performed on the validation part.

%   Multivariate Gaussian Classifier

\subsubsection{Multivariate Gaussian Classifier}
\label{subsubsec:multivariateGaussianClassifier}
The first classifier is MVG and it is given by the empirical mean and covariance matrix for each class,
\begin{equation}
    \mu_c^* = \frac{1}{N_c} \sum_{i\mid c_i=c} x_i\text{ ,}\quad
    \Sigma_c^* = \frac{1}{N_c} \sum_{i\mid c_i=c} (x_i - \mu_c^*)(x_i - \mu_c^*)^T
    \label{eq:meanAndVarianceMVG}
\end{equation}

%   Naive Bayes Gaussian Classifier

\subsubsection{Naive Bayes Gaussian Classifier}
\label{subsubsec:naiveBayesGaussianClassifier}
This model makes an important assumption that simplifies the number of parameters to be estimated,
it assumes that the features are independent given their class.
This causes the covariance matrix to be a diagonal matrix, consequently, matching MVG with a diagonal covariance matrix.
However, the assumption of independence may be too restrictive and lead to inferior performance if the features are indeed correlated.

\begin{equation}
    \mu_{c,[j]}^* = \frac{1}{N_c} \sum_{i\mid c_i = c} x_{i,[j]}\text{ ,}\quad
    \sigma_{c,[j]}^2 = \frac{1}{N_c} \sum_{i\mid c_i = c} (x_{i,[j]} - \mu_{c,[j]}^*)^2
    \label{eq:meanAndVarianceNBG}
\end{equation}

%   Tied Covariance Gaussian Classifier

\subsubsection{Tied Covariance Gaussian Classifier}
\label{subsubsec:tiedCovarianceGaussianClassifier}
The assumption of the latter model consists of its own average for each class, but an equal covariance matrix for all classes.

\begin{equation}
    \mu_c^* = \frac{1}{N_c} \sum_{i\mid c_i=c} x_i\text{ ,}\quad
    \Sigma^* = \frac{1}{N} \sum_{c} \sum_{i\mid c_i = c} (x_{i} - \mu_c)(x_{i} - \mu_c)^T
    \label{eq:tiedCovariance}
\end{equation}

The characteristic of this model is that it is strongly correlated to LDA.

\subsubsection{Gaussian Models Comparison}
\label{subsubsec:gaussianModelsComparison}
A threshold of \(0\) was used to perform our results, which means that \(P(C=1)=P(C=0)=1/2\). This model was applied and
the outcomes can be seen in the \autoref{tab:resultGaussianClassificationModels}.

\begin{table}[h]
    \centering
    \begin{tabular}{c c c}
        \toprule
        \textbf{Features} & \textbf{Model}  & \textbf{Error Rate} (\%) \\
        \midrule
        \multicolumn{3}{c}{\textit{no PCA}} \\
        \midrule
        1 to 6            & MVG             & 7.00                     \\
        1 to 6            & Naive Bayes     & 7.20                     \\
        1 to 6            & Tied Covariance & 9.30                     \\
        \midrule
        1 to 4            & MVG             & 7.95                     \\
        1 to 4            & Naive Bayes     & 7.65                     \\
        1 to 4            & Tied Covariance & 9.50                     \\
        \midrule
        1 - 2             & MVG             & 36.50                    \\
        1 - 2             & Naive Bayes     & 36.30                    \\
        1 - 2             & Tied Covariance & 49.45                    \\
        \midrule
        3 - 4             & MVG             & 9.45                     \\
        3 - 4             & Naive Bayes     & 9.45                     \\
        3 - 4             & Tied Covariance & 9.40                     \\
        \midrule
        \multicolumn{3}{c}{\textit{PCA m = 5}} \\
        \midrule
        1 to 6            & MVG             & 7.10                     \\
        1 to 6            & Naive Bayes     & 8.75                     \\
        1 to 6            & Tied Covariance & 9.30                     \\
        \midrule
        \multicolumn{3}{c}{\textit{PCA m = 6}} \\
        \midrule
        1 to 6            & MVG             & 7.00                     \\
        1 to 6            & Naive Bayes     & 8.90                     \\
        1 to 6            & Tied Covariance & 9.30                     \\
        \bottomrule
    \end{tabular}
    \captionsetup{justification=justified,singlelinecheck=false,format=hang}
    \caption{Table showing the results of the Error Rate for different Models and Features.}
    \label{tab:resultGaussianClassificationModels}
\end{table}

Comparing the results with the \autoref{tab:LDAPCAForClassification}, we can see that for some configurations
there were improvements in terms of error rate.
This means that Gaussian models are better able to classify the data.
If we go into the details of how the error rate changes as a function of the observed features,
we can see that:
\begin{itemize}
    \item \textbf{1 to 6}: in the case we consider all 6 features, the error rate is quite low and its range
    goes from 7.00\% to 9.30\%. This means that they all provide useful information.
    \item \textbf{1 to 4}: if we consider features from 1 to 4, we can see that the error rate increases slightly.
    This allow us to say that features 5 and 6 have useful but not fundamental information to change the outcome.
    \item \textbf{1 - 2}: features 1 and 2 have a rather high error rate, meaning that they don't contain
    relevant information.
    \item \textbf{3 - 4}: on the other hand, the latter two features considered have a rather low error rate, a value
    close to the case where all features are considered. This means that the information contained in these two
    features is relevant for making the classification.
\end{itemize}


